%#! platex  -halt-on-error -file-line-error -src-specials main.tex
% Time-stamp: <2010-10-27 11:10:29 takago>

\appendix %付録
\chapter{あああああ}
ここにプログラムの使い方,セットアップ方法などを書きましょう.
ここにプログラムの使い方,セットアップ方法などを書きましょう.ここにプログラここにプログラムの使い方,セットアップ方法などを書きましょう.ムの使い方,セットアップ方法などを書きましょう.
ここにプログラムの使い方,セットアップ方法などを書きましょう.ここにプログラムの使い方,セットアップ方法などを書きましょう.
%%%%%%%%%%%%% プログラムの埋め込み %%%%%%%%%%%%%%%%%%%%%%%%%

%% ファイル名を指定して、挿入する場合
\lstinputlisting[language=c,caption=サンプルプログラム,label=sample.c]{appendix/src/sample.c}

%% 直接プログラムを埋め込む場合
\begin{lstlisting}[language=ruby,caption=スパゲッティソース,label=test.rb]
#! /usr/local/bin/ruby -Ks
# numbers.rb
print "正の整数値を表す文字列を入力してください。正の整数値を表す文字列を入力してください。\n"
while true
	print ">"
	line = gets.chomp # 改行コードを切り捨てる
	break if line.empty?
	begin
		v = Integer(line) # 文字列を整数化する
	rescue
		puts "変換できません。"
		next
	end
	printf ("2進法:%b\n",v)
	printf ("8進法:%o\n",v)
	printf ("10進法:%d\n",v)
	printf ("16進法:%x\n",v)
end
puts "Bye."
\end{lstlisting}
\chapter{いいいいい}
