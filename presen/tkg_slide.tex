%
% このファイルは書き換えないで下さい
%
\documentclass[25pt, landscape,dvipdfmx, oneside, uplatex]{foils}

% 4:3のスクリーン(8号館)
\usepackage[top=15truemm,bottom=22truemm,left=15truemm,right=15truemm,paperwidth=300truemm,paperheight=225truemm]{geometry}

% 16:9のスクリーン(23号館)
%\usepackage[top=15truemm,bottom=22truemm,left=15truemm,right=15truemm,paperwidth=320truemm,paperheight=180truemm]{geometry}
\setlength{\foilheadskip}{0mm}
\setlength{\parindent}{0mm}

\usepackage[deluxe]{otf}
\usepackage[T1]{fontenc}
\usepackage{lmodern}
\renewcommand{\kanjifamilydefault}{\gtdefault}  % 日本語フォントをゴシックに
\mathversion{bold} % 数式フォントを太字に変更する


\usepackage{bbding} % \PencilRightDown 鉛筆マーク
\usepackage{tikz}
\usepackage[symbol]{footmisc}
\usepackage{xcolor,listings, plistings}
\usepackage{multicol}
\usepackage{url}
\usepackage{graphicx}
\usepackage{epsfig}

\usepackage{enumerate}
\usepackage{lastpage}
\usepackage{ascmac}
\usepackage{fancybox,fancyvrb,okumacro}

\renewcommand{\lstlistingname}{List} 
\usepackage[%
pdfstartview={FitH -32768},%    描画領域の幅に合わせる
bookmarks=true,%                しおり付き
bookmarksnumbered=true,%        章や節の番号をふる
bookmarkstype=toc,%             目次情報のファイル.tocを参照
colorlinks=true,%              ハイパーリンクを色枠に
linkcolor=black,%
urlcolor=black,%
citecolor=black,%
filecolor=black,%
menucolor=black,%
pagecolor=black,%
]{hyperref}
\lstset{%
	language={Python}, 
	backgroundcolor={\color[gray]{.95}},%
	basicstyle={\ttfamily\small},%
	identifierstyle={\ttfamily\small},
	commentstyle={\ttfamily\small\color{red}},
	keywordstyle={\ttfamily\small\color{blue}},
	ndkeywordstyle={},%
	stringstyle={\ttfamily\small\color[rgb]{0,0.5,0}},
	frame={tb},
	breaklines=true,
    columns=[l]{fullflexible},
%	columns=[l]{fixed},% fixed だと開きすぎ
	basewidth=0.5em,   % これないと行頭のスペースが揃わない
	numbers=left,%
	xrightmargin=0zw,%
	xleftmargin=3zw,%
	numberstyle={\ttfamily\small},%
	tabsize=3,
	stepnumber=1, 
	numbersep=0.75zw,%
	lineskip=-0.3ex,%
	belowcaptionskip=3pt,   % これないと見出しとリスト本体に隙間が毎回変わる
	abovecaptionskip=0pt,
	captionpos=t,
	showstringspaces=false, % 半角スペースを記号で表示しない
}
%

%%%%%%%%%%%% fboxの線幅
\setlength{\fboxrule}{1.5pt}



%%%%%%%%%%%% ページフッタの設定 %%%%%%%%%%%%%
\usepackage{fancyhdr}
\pagestyle{fancy}
\fancyhf{}  % これを入れないと2ページのフッタがずれる
\renewcommand{\headrulewidth}{0pt} % 水平線を消去
\renewcommand{\footrulewidth}{0pt} % 水平線を消去
\renewcommand{\thefootnote}{\fnsymbol{footnote}}
